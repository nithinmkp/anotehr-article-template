% Options for packages loaded elsewhere
\PassOptionsToPackage{unicode}{hyperref}
\PassOptionsToPackage{hyphens}{url}
%
\documentclass[
  a4paper,
  abstract]{scrartcl}

\usepackage{amsmath,amssymb}
\usepackage{setspace}
\usepackage{iftex}
\ifPDFTeX
  \usepackage[T1]{fontenc}
  \usepackage[utf8]{inputenc}
  \usepackage{textcomp} % provide euro and other symbols
\else % if luatex or xetex
  \usepackage{unicode-math}
  \defaultfontfeatures{Scale=MatchLowercase}
  \defaultfontfeatures[\rmfamily]{Ligatures=TeX,Scale=1}
\fi
\usepackage{lmodern}
\ifPDFTeX\else  
    % xetex/luatex font selection
\fi
% Use upquote if available, for straight quotes in verbatim environments
\IfFileExists{upquote.sty}{\usepackage{upquote}}{}
\IfFileExists{microtype.sty}{% use microtype if available
  \usepackage[]{microtype}
  \UseMicrotypeSet[protrusion]{basicmath} % disable protrusion for tt fonts
}{}
\makeatletter
\@ifundefined{KOMAClassName}{% if non-KOMA class
  \IfFileExists{parskip.sty}{%
    \usepackage{parskip}
  }{% else
    \setlength{\parindent}{0pt}
    \setlength{\parskip}{6pt plus 2pt minus 1pt}}
}{% if KOMA class
  \KOMAoptions{parskip=half}}
\makeatother
\usepackage{xcolor}
\setlength{\emergencystretch}{3em} % prevent overfull lines
\setcounter{secnumdepth}{5}
% Make \paragraph and \subparagraph free-standing
\makeatletter
\ifx\paragraph\undefined\else
  \let\oldparagraph\paragraph
  \renewcommand{\paragraph}{
    \@ifstar
      \xxxParagraphStar
      \xxxParagraphNoStar
  }
  \newcommand{\xxxParagraphStar}[1]{\oldparagraph*{#1}\mbox{}}
  \newcommand{\xxxParagraphNoStar}[1]{\oldparagraph{#1}\mbox{}}
\fi
\ifx\subparagraph\undefined\else
  \let\oldsubparagraph\subparagraph
  \renewcommand{\subparagraph}{
    \@ifstar
      \xxxSubParagraphStar
      \xxxSubParagraphNoStar
  }
  \newcommand{\xxxSubParagraphStar}[1]{\oldsubparagraph*{#1}\mbox{}}
  \newcommand{\xxxSubParagraphNoStar}[1]{\oldsubparagraph{#1}\mbox{}}
\fi
\makeatother


\providecommand{\tightlist}{%
  \setlength{\itemsep}{0pt}\setlength{\parskip}{0pt}}\usepackage{longtable,booktabs,array}
\usepackage{calc} % for calculating minipage widths
% Correct order of tables after \paragraph or \subparagraph
\usepackage{etoolbox}
\makeatletter
\patchcmd\longtable{\par}{\if@noskipsec\mbox{}\fi\par}{}{}
\makeatother
% Allow footnotes in longtable head/foot
\IfFileExists{footnotehyper.sty}{\usepackage{footnotehyper}}{\usepackage{footnote}}
\makesavenoteenv{longtable}
\usepackage{graphicx}
\makeatletter
\def\maxwidth{\ifdim\Gin@nat@width>\linewidth\linewidth\else\Gin@nat@width\fi}
\def\maxheight{\ifdim\Gin@nat@height>\textheight\textheight\else\Gin@nat@height\fi}
\makeatother
% Scale images if necessary, so that they will not overflow the page
% margins by default, and it is still possible to overwrite the defaults
% using explicit options in \includegraphics[width, height, ...]{}
\setkeys{Gin}{width=\maxwidth,height=\maxheight,keepaspectratio}
% Set default figure placement to htbp
\makeatletter
\def\fps@figure{htbp}
\makeatother
% definitions for citeproc citations
\NewDocumentCommand\citeproctext{}{}
\NewDocumentCommand\citeproc{mm}{%
  \begingroup\def\citeproctext{#2}\cite{#1}\endgroup}
\makeatletter
 % allow citations to break across lines
 \let\@cite@ofmt\@firstofone
 % avoid brackets around text for \cite:
 \def\@biblabel#1{}
 \def\@cite#1#2{{#1\if@tempswa , #2\fi}}
\makeatother
\newlength{\cslhangindent}
\setlength{\cslhangindent}{1.5em}
\newlength{\csllabelwidth}
\setlength{\csllabelwidth}{3em}
\newenvironment{CSLReferences}[2] % #1 hanging-indent, #2 entry-spacing
 {\begin{list}{}{%
  \setlength{\itemindent}{0pt}
  \setlength{\leftmargin}{0pt}
  \setlength{\parsep}{0pt}
  % turn on hanging indent if param 1 is 1
  \ifodd #1
   \setlength{\leftmargin}{\cslhangindent}
   \setlength{\itemindent}{-1\cslhangindent}
  \fi
  % set entry spacing
  \setlength{\itemsep}{#2\baselineskip}}}
 {\end{list}}
\usepackage{calc}
\newcommand{\CSLBlock}[1]{\hfill\break\parbox[t]{\linewidth}{\strut\ignorespaces#1\strut}}
\newcommand{\CSLLeftMargin}[1]{\parbox[t]{\csllabelwidth}{\strut#1\strut}}
\newcommand{\CSLRightInline}[1]{\parbox[t]{\linewidth - \csllabelwidth}{\strut#1\strut}}
\newcommand{\CSLIndent}[1]{\hspace{\cslhangindent}#1}

\usepackage{hyperref}
\usepackage{cleveref}
\crefformat{equation}{equation~(#2#1#3)}
\Crefformat{equation}{Equation~(#2#1#3)}
%\renewcommand{\theequation}{\roman{equation}}
%\setkomafont{disposition}{\bfseries}
\usepackage[noblocks]{authblk}
%\usepackage{etoolbox}
\renewcommand*{\Authsep}{, }
\renewcommand*{\Authand}{, }
\renewcommand*{\Authands}{, }
\renewcommand\Affilfont{\small}
%\setkomafont{title}{\normalfont\bfseries}
%\makeatletter
%\patchcmd{\@maketitle}{\titlefont\small}{\titlefont\tiny}{}{}
%\makeatother
\usepackage[c2]{optidef}
\usepackage{booktabs, longtable, colortbl, array}
\usepackage[flushleft]{threeparttable}
\usepackage{tabularray}
\usepackage[singlelinecheck=false ]{caption}
\Crefformat{table}{Table~(#2#1#3)}
\Crefformat{figure}{Figure~(#2#1#3)} 
\usepackage{float}   

\Crefrangeformat{table}{Tables~(#3#1#4)~to~(#5#2#6)}
\Crefrangeformat{figure}{Figures~(#3#1#4)~to~(#5#2#6)}

\Crefmultiformat{table}{Table~(#2#1#3)}%
{ and~(#2#1#3)}% the second argument is what comes after the first reference
{, (#2#1#3)}% for three or more, separator format
{ and~(#2#1#3)}% for the final reference in a series

\Crefmultiformat{figure}{Figure~(#2#1#3)}%
{ and~(#2#1#3)}% the second argument is what comes after the first reference
{, (#2#1#3)}% for three or more, separator format
{ and~(#2#1#3)}% for the final reference in a series

%\usepackage{fancyhdr}
%\pagestyle{fancy}
%\fancyhead{}
%\fancyhead[RO,LE]{\textbf{Consumer Sentiments}}

%\usepackage{enotez}
%\let\footnote=\endnote
%\setkomafont{title}{\small}

\makeatletter
\@ifpackageloaded{caption}{}{\usepackage{caption}}
\AtBeginDocument{%
\ifdefined\contentsname
  \renewcommand*\contentsname{Table of contents}
\else
  \newcommand\contentsname{Table of contents}
\fi
\ifdefined\listfigurename
  \renewcommand*\listfigurename{List of Figures}
\else
  \newcommand\listfigurename{List of Figures}
\fi
\ifdefined\listtablename
  \renewcommand*\listtablename{List of Tables}
\else
  \newcommand\listtablename{List of Tables}
\fi
\ifdefined\figurename
  \renewcommand*\figurename{Figure}
\else
  \newcommand\figurename{Figure}
\fi
\ifdefined\tablename
  \renewcommand*\tablename{Table}
\else
  \newcommand\tablename{Table}
\fi
}
\@ifpackageloaded{float}{}{\usepackage{float}}
\floatstyle{ruled}
\@ifundefined{c@chapter}{\newfloat{codelisting}{h}{lop}}{\newfloat{codelisting}{h}{lop}[chapter]}
\floatname{codelisting}{Listing}
\newcommand*\listoflistings{\listof{codelisting}{List of Listings}}
\makeatother
\makeatletter
\makeatother
\makeatletter
\@ifpackageloaded{caption}{}{\usepackage{caption}}
\@ifpackageloaded{subcaption}{}{\usepackage{subcaption}}
\makeatother

\ifLuaTeX
  \usepackage{selnolig}  % disable illegal ligatures
\fi
\usepackage{bookmark}

\IfFileExists{xurl.sty}{\usepackage{xurl}}{} % add URL line breaks if available
\urlstyle{same} % disable monospaced font for URLs
\hypersetup{
  pdftitle={Consumer Sentiments and its determinants},
  pdfauthor={Nithin.M; XYZ; pqr},
  pdfkeywords={one, two},
  hidelinks,
  pdfcreator={LaTeX via pandoc}}


\title{Consumer Sentiments and its determinants\thanks{Authors would
like to thank Ms.~Monami Mitra, Director, Price Statistics Division,
National Statistical Office, Government of India}}

\author[1]{Nithin.M%
\thanks{Corresponding author.\\ Email: \texttt{write2nithinm@iitkgp.ac.in}}%  % Add corresponding author footnote if applicable
   }
\author[2]{XYZ%
%  % Add corresponding author footnote if applicable
   }
\author[1, 2]{pqr%
%  % Add corresponding author footnote if applicable
   }

\affil[1]{
      Department of Humanities and Social Sciences, IIT Kharagpur
  }
\affil[2]{
      University of Hyderabad
  }

\date{}
\begin{document}
\maketitle

\begin{abstract}
\setstretch{1.2}
The extant literature have already examined the excess sensitivity of
consumption, and the validity of the permanent income and life cycle
hypothesis (PIH) for the developed countries. India is a big country
with a significantly large cross sectional heterogeneity than the
developed countries. It provides a rich information content to the micro
level Indian data, which is nicely captured in the longitudinal dataset
for the Indian households, given by CPHS, CMIE. Using this rich dataset,
and inspired from Souleles (2004), this paper is the first genuine
attempt to examine the excess sensitivity of consumption to sentiments,
and the validity of PIH for India through an Euler equation framework.
We find - (i) the excess sensitivity of consumption to sentiments
exists, and PIH does not hold for India, (ii) since, sentiments mostly
captures the perception household's uncertainty about their own
financial condition, and the overall business conditions, precautionary
savings motive holds for the Indian households, and (iii) the excess
sensitivity of consumption to sentiments exists even after controlling
for the household specific forecast errors about their own financial
conditions. %
\\
\\
\noindent%
\textbf{Keywords:} one, two %
 
\noindent%
\textbf{JEL:} one, two
\end{abstract}


\newpage

\setstretch{1.2}
\section{Introduction}\label{sec-1}

Domestic household consumption is an important macroeconomic aggregate
and accounts for about 60 per cent of national expenditure\footnote{\href{https://pib.gov.in/PressReleasePage.aspx?PRID=2010223}{See
  MSOPI}}. Consumer Sentiments contain information about potential
future changes in consumer spending and hence is of paramaount
importance in business cycle research
(\citeproc{ref-curtin_consumer_2007}{Curtin, 2007};
\citeproc{ref-katona1951}{Katona, 1951}). Consumer sentiments is an
important indicator of business conditions of the economy and is often
closely monitored by researchers and policy makers alike. Ever since the
time series data on consumer sentiments have become available, here has
been a prolonged interest in the power of consumer sentiment to predict
business cycle fluctuations, and aggregate consumption growth
(\citeproc{ref-matsusaka_consumer_1995}{Matsusaka \& Sbordone, 1995};
\citeproc{ref-mishkin_consumer_1978}{Mishkin et al., 1978}). Using US
data from 1953-1988, Matsusaka \& Sbordone
(\citeproc{ref-matsusaka_consumer_1995}{1995}) finds that all recesions
were preceeded by a fall in confidence and all major falls in consumer
sentiments were followed by a recession. It can be thought of as
consumers correctly forecasting a fall in output or a fall in consumer
confidence causing a fall in output. Despite the widespread attention
given to surveys of consumer confidence, the mechanisms by which
household attitudes influence the real economy are less well understood
(\citeproc{ref-ludvigson_consumer_2004}{Ludvigson, 2004}).

\section*{References}\label{references}
\addcontentsline{toc}{section}{References}

\phantomsection\label{refs}
\begin{CSLReferences}{1}{0}
\bibitem[\citeproctext]{ref-curtin_consumer_2007}
Curtin, R. (2007). Consumer {Sentiment} {Surveys}: {Worldwide} {Review}
and {Assessment}. \emph{Journal of Business Cycle Measurement and
Analysis}, \emph{2007}(1), 7--42.
\url{https://doi.org/10.1787/jbcma-v2007-art2-en}

\bibitem[\citeproctext]{ref-katona1951}
Katona, G. (1951). \emph{Psychological analysis of economic behavior.}

\bibitem[\citeproctext]{ref-ludvigson_consumer_2004}
Ludvigson, S. C. (2004). Consumer {Confidence} and {Consumer}
{Spending}. \emph{Journal of Economic Perspectives}, \emph{18}(2),
29--50. \url{https://doi.org/10.1257/0895330041371222}

\bibitem[\citeproctext]{ref-matsusaka_consumer_1995}
Matsusaka, J. G., \& Sbordone, A. M. (1995). {CONSUMER} {CONFIDENCE}
{AND} {ECONOMIC} {FLUCTUATIONS}. \emph{Economic Inquiry}, \emph{33}(2),
296--318. \url{https://doi.org/10.1111/j.1465-7295.1995.tb01864.x}

\bibitem[\citeproctext]{ref-mishkin_consumer_1978}
Mishkin, F. S., Hall, R., Shoven, J., Juster, T., \& Lovell, M. (1978).
Consumer {Sentiment} and {Spending} on {Durable} {Goods}.
\emph{Brookings Papers on Economic Activity}, \emph{1978}(1), 217.
\url{https://doi.org/10.2307/2534366}

\end{CSLReferences}




\end{document}
